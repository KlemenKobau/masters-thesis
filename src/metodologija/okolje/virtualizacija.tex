\subsubsection{Virtualizacija}

V tem poglavju se odločimo,
ali bomo uporabljali virtualizacijo,
ali nameščali aplikacijo neposredno na operacijski sistem.
Članek~\cite{virtualization_survey} predstavi prednosti in slabosti virtualizacije,
kar predstavim v nadaljevanju.

\bigskip

\noindent Prednosti:
\begin{itemize}
    \item Prilagodljivost, virtualne naprave lahko prenašamo 
    iz enega računalnika na drugega.
    So neodvisne od strojne opreme.

    \item Razpoložljivost, v primeru ustavitve strojne opreme,
    lahko virtualno napravo prestavimo na drug računalnik in
    naprava tam normalno deluje naprej.

    \item Skalabilnost, nove navidezne naprave enostavno dodamo.
    Z dodajanjem strojne opreme navidezne naprave enostavno prenesemo nanjo.

    \item Izraba strojne opreme, navidezne naprave porabijo spomin,
    ki ga gostitelj ne uporablja.

    \item Varnost, stroitve so izolirane ena od druge
    \item Cena, več manjših strežnikov lahko nadomestimo z enim večjim,
    kar ponavadi skupno zniža ceno.
    \item Prilagodljivost na zahtevnost dela, navideznim napravam 
    lahko dodelimo samo toliko virov, kot jih potrebuje
    \item Prerazporeditev obremenitve, v primeru visoke obremenitve na
    enem računalniku, lahko navidezne naprave migriramo na drugega in
    tako prerazporedimo obremenitev
    \item Podpora za storitve, ki zahtevajo določen operacijski sistem
\end{itemize}

\noindent Slabosti:
\begin{itemize}
    \item Virtualizacija pobere sistemske vire že sama od sebe
    \item Še vedno obstaja odvisnost od strojne opreme
    (v primeru izpada elektrike izgubimo tudi dostop do navidezne naprave).
    \item Ko izberemo orodje za virtualizacijo, smo nanj vezani.
    Migracija ni vedno enostavna.
    \item Cena licenc orodij za virtualizacijo.
\end{itemize}

\noindent Glede na napisane prednosti in slabosti se odločimo,
ali želimo uporabljati virtualne naprave ali ne.
V poglavju~\ref{subsub:oblak-on-prem} smo se odločali med oblakom in
uporabo lastne strojne opreme.

V primeru, da smo se odločili za lastno strojno opremo,
se moramo odločiti za vrsto hipervizorja,
ki jo želimo uporabljati.
Obstaja več vrst hipervizorjev in vsaka ima svoje prednosti
in slabosti, ki pa so precej tehnične in izven obsega te magistrske naloge.

V primeru, da smo se odločili za oblak,
se moramo odločati med najemom virtualk ali najemov samih fizičnih strojev.
V primeru najema fizičnih strojev postane naša izbira podobna izbiri
lastne strojne opreme in moramo hipervizor namestiti sami.
V primeru najema virtualk, pa nam ponudnik oblaka sam zagotovi
virtualke in nam tako v tem koraku ni potrebno narediti nič dodatnega.

V naslednjem poglavju se bomo odločali o uporabi vsebnikov in
orodij za orkestracijo.
Vsebniki imajo podobne prednosti in slabosti kot
virtualke, ampak na nivoju aplikacije.
Uporaba virtualk na izbiro uporabe orkestracije ne vpliva,
med tem ko ima izbira oblaka podoben vpliv kot pri
izbiri virtualizacije.