\subsubsection{Orkestracija}

V tem poglavju se odločamo o izbiri orkestracije ali
ročnega nadziranja aplikacij.
Orodja za orkestracijo skrbijo za nemoteno delovanje aplikacij.
V primerjavi z ročnim upravljanjem orodje za orkestracijo samo skrbi za:
\begin{itemize}
    \item Ugašanje aplikacij, ki so obtičale.
    \item Povezovanjem med aplikacijami.
    \item Zunanjo shrambo.
    \item Skaliranje aplikacij.
\end{itemize}

\noindent Članek~\cite{virtualisation_vs_containerization} primerja uporabo
vsebnikov z uporabo navideznih naprav in primerja
nazlične implementacije orodij za delo z vsebniki.
Moderne aplikacije so sestavljene iz mnogih storitev, zato se za
lažje delo z vsebniki priporoča uporaba orodij za orkestracijo~\cite{container_orchestration}.
Članek~\cite{container_orchestrators_comparison} primerja različne implementacije.
Avtorji priporočajo uporabo Docker Compose ali Docker swarma za testiranje,
ter uporabo Kubernetesa za produkcijo.
Članek~\cite{container_orchestration} primerja implementacije bolj podrobno in
prav tako predlaga Kubernetes,
ampak opozarja na visoko porabo virov v določenih primerih.
Oba članka kot eno izmed možnosti omenjata tudi Apache Mesos,
ki pa ima manjšo bazo uporabnikov in je bolj zahteven za postavitev.
Za primere testiranja predlagamo uporabo orodja Minikube,
ki omogoča enostavno postavitev testne Kubernetes gruče.

Nadaljevanje magistrske naloge se nagiba k uporabi orkestracije,
saj z rastjo arhitekture postane ročno uporavljanje z aplikacijami preveč
zapleteno.
V primeru, da se odločite za ročno upravljanje je potrebno vsak del
arhitekture postaviti po navodilih razvijalca.
Vključitev v magistrsko nalogo je preveč zahtevna,
saj nameščanje vsebuje preveč spremenljivk, na primer 
od operacijskega sistema in tipa podatkovnega sistema.

V primeru, da smo se v poglavju~\ref{subsub:oblak-on-prem} odločili za uporabo oblaka,
nam naš ponudnik pogosto ponuja tudi že uporabo njhovega
orodja za orkestracijo.
V tem primeru, nam v tem koraku ni potrebno narediti ničesar,
drugače pa moramo orodje za orkestracijo postaviti sami.
Če smo izbrali orodje Kubernetes, si lahko pomagamo z orodji \textit{Kubespray} in \textit{kubeadm}
za lažjo postavitiv Kubernetesa.

\bigskip

\noindent V naslednjem poglavju bomo zgradili arhitekturo okoli 