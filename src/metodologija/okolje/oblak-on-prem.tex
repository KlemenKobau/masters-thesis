\subsubsection{Uporaba lastne strojne opreme ali oblaka} \label{subsub:oblak-on-prem}

Pri izbiri okolja, se moramo najprej odločiti, ali bomo uporabljali lastne strežnike ali
storitve v oblaku.
Članki~\cite{cloud_computing_issues_challanges,cloud_computing_overview,cloud_computing_today_tomorrow}
predstavijo prednosti in slabosti infrastrukture v oblaku.
V tabeli~\ref{tab:cloud-vs-local-infrastructure} predstavim povzetek primerjave uporabe oblačne in
lastne infrastrukture.

\begin{table}[H]
    \centering
    \begin{tabularx}{\textwidth}{S L L} 
        & Lokalno & Oblak \\
        \hline
        
        Skaliranje & 
        Se lahko prilagodi, 
        ampak stroškov nakupa strežnikov ne povrnemo & 
        Se prilagaja zahtevam \\

        Zasebnost podatkov &
        Podatki ne zapustijo naših strežnikov &
        Podatki se pošiljajo preko omrežja,
        lahko v tuje države,
        na tuje sisteme \\

        Stroški &
        Skupni stroški infrastrukture so višji, 
        stroški posamezne enote procesiranja (npr. ene virtualne naprave), so nižji &
        Skupni stroški infrastrukture so nižji, stroški prenosa podatkov so višji \\
        
        Migracija &
        Migracija med lastnimi strežniki je enostavna &
        Migracija med ponudniki je težka \\

        Dostopnost &
        Ponavadi nižja &
        Ponavadi višja
    \end{tabularx}
    
    \caption{Primerjava infrastrukture v oblaku in lokalno.}
    \label{tab:cloud-vs-local-infrastructure}
\end{table}

\noindent Tabela~\ref*{tab:cloud-vs-local-infrastructure} nam pomaga pri izbiri
okolja.
V naslednjih odsekih bodo naše odločitve odvisne od izbire v tem poglavju.
Takrat se bomo sklicevali na to poglavje in razložili,
na kaj moramo biti pazljivi in kakšne so razlike zaradi odločitve
v tem odseku.