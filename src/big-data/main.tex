\section{Big data} \label{sec:big-data}

Definicija Big data še ni ustaljena.
Članek~\cite{toward_scalable_systems_big_data_analytics} predstavi 3 definicije.

\paragraph{Definicija preko lastnosti}
Big data aplikacije obdelujejo podatke z naslednjimi lastnostmi, katerimi se
skupno reče 4V, po angleških besedah za lastnosti.
Članek~\cite{modeling_requirements_big_data} navede tu še nekaj razlogov za izbrane
lastnosti\footnote{Podatki so iz leta 2013, številke so se samo še povečale.}.

\begin{itemize} \label{list:four_v}
    \item Volumen podatkov (\emph{Volume}) -- Aplikacija uporablja velike količine podatkov (več terabajtov, petabajtov, zetabitov)
          \begin{itemize}
              \item Članek~\cite{modeling_requirements_big_data} napoveduje, da bo do leta 2020 generiranih 40 zetabitov podatkov
                    \footnote{Vir~\cite{big_data_amount_statista} to številko poveča na 64 zetabitov}, kar je 300 kratno povečanje od leta 2005
              \item 6 od 7 biljonov ljudi ima mobilni telefon
              \item Večina podjetij v ZDA ima vsaj 100 terabitov shranjenih podatkov
          \end{itemize}

    \item Hitrost podatkovnih tokov (\emph{Velocity}) -- Podatki so generirani v realnem času, ali skoraj v realnem času
          \begin{itemize}
              \item Newyorška borza shrani 1 terabit podatkov med vsako trgovsko sejo
              \item Moderni avtomobili imajo skoraj 100 senzorjev
          \end{itemize}

    \item Raznolikost podatkov (\emph{Variety}) -- podatkovni modeli so manj strokturirani, imamo več virov podatkov
          \begin{itemize}
              \item Od leta 2011 je bila ocena velikosti podatkov okoli 150 Exabitov
              \item Uporabniki YouTube pogledajo več kot 4 miljarde ur videoposnetkov vsak mesec
              \item 400 miljonov tweetov se pošlje vsak dan
          \end{itemize}

    \item Zaupanje v podatke (\emph{Veracity}) -- podatki so lahko napačni, napake v podatkih.
          \begin{itemize}
              \item 1 izmed 3 vodilnih podjetij ne zaupa podatkom, katere uporabijo za odločitve
          \end{itemize}
\end{itemize}

\paragraph{Definicija preko primerjave}
Big data pomeni delo s količino podatkov, ki jo tradicionalne baze ne podpirajo.

\paragraph{Definicija preko arhitekture}
Big data pomeni delo s podatki, kjer njihova količina preprečuje uporabo
tradicionalnih baz in potrebujemo precejšnje horizontalno skaliranje.

\bigskip

\noindent Big data nima ustaljene definicije, ampak s prej napisanimi si lahko pomagamo pri razumevanju razlik
pri delu z big data in tradicionalnim delom s podatki.

\subsection{Primerjava s tradicionalnimi podatki}

Članek~\cite{toward_scalable_systems_big_data_analytics} primerja delo s tradicionalnimi podatki in z big data.
Iz razlik med vrstama lažje razumemo zakaj tradicionalne metode za obdelavo podatkov niso primerne za obdelavo big data podatkov.

\begin{table}[H]
    \centering
    \begin{tabular}{l l l}
                            & Tradicionalni  & Big Data                     \\
        \hline
        Volumen             & GB             & se povečuje (TB, PB, EB)     \\
        Hitrost generiranja & vsako uro, dan & hitreje kot tradicionalni    \\
        Struktura           & strukturirani  & delno ali nestrukturirani    \\
        Viri                & centralizirani & razpršeni                    \\
        Integriteta         & enostavna      & težka                        \\
        Shramba             & relacijska     & HDFS ali NoSQL               \\
        Dostop              & interaktiven   & v sklopih ali v realnem času
    \end{tabular}
    \caption{Primerjava podatkov v big data in v navadnih aplikacijah. Tabela je prevedena iz~\cite{toward_scalable_systems_big_data_analytics}.}
    \label{table:big-data-vs-normal}
\end{table}

\subsection{Izzivi in priložnosti Big Data}

\todo{Napiši kaj o izzivih in priložnostih. O tem imam nek članek.}

\subsection{Primeri big data aplikacij}
Avtorji~\cite{vehicle_networks} predstavijo način izboljšave prenosa podatkov med vozili.
Uporabijo povezave naprava-naprava, kjer skupine avtomobilov skupaj tvorijo center za oddajo podatkov, ki
so zelo zahtevani in je zahtevkov preveč za eno samo vozilo.
Soočajo se z velikimi količinami podatkov, kjer kjer lahko deli arhitekture nenadoma odpovejo.
Podatki so nestrukturirani in generirani v realnem času.

Članek~\cite{a_big_data_analytics_based_methodology} govori o uporabi big data za sprejemanje strateških odločitev
pri trgovanju.
Avtorji nadgradijo CRISP-DM\todo{dodaj v slovarček, opiši, citiraj} metodologijo.
Izboljšano metodologijo nato uporabijo za analizo trgov.

Avtorji~\cite{scalable_framework_for_provisioning_large_scale_iot_deployments} predstavijo orodje LEONORE,
ki omogoča elastično postavljanje ogrodja za aplikacije na meji\todo{edge computing, dodaj prevod}.
Soočijo se s težavo postavljanja procesorsko zahtevnih aplikacij na napravah z malo procesorske moči.
\improvement{Dopolni}

Več primerov big data aplikacij na različnih področjih si lahko preberete v~\cite{big_data_analytics_societal_impact}.

\todo{Dodaj več primerov}