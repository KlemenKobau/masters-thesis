\section{Uvod}

Informacijski sistemi, ki ji jih vsakodnevno uporablja sodobna družba, 
generirajo velike količine podatkov. 
Podatke generirajo uporabniki in aplikacije ...same. Ker je shranjevanje in 
obdelava velike količne podatkov podjetjem dostopnejša kot je bila nekoč, 
podjetja le te zkoriščajo za različne namene, od izboljšanja poslovnih strategij, 
dviga dodane vrednosti, segmentacije in profiliranja kupcev, 
do izboljšanja delovanja samih informacijskih sistemov in preprečevanja napak.
Za obdelavo velikih količin podatkov se je razvila računalniška smer, 
\textit{Podatkovne vede}, ki se ukvarja z analizo
in obvladovanjem teh podatkov.
Z \enquote{veliko količino} mislimo količine, ki segajo v terabite ali petabite 
in so generirane v sekundnih intervalih~\cite{toward_scalable_systems_big_data_analytics}.
Tradicionalne metode niso kos takim podatkovnim tokom, zato za obdelavo velike količine podatkov, 
predvsem v realnem času, 
potrebujemo specializirane tehnologije.
Takšni sistemi za obvladovanje kompleksnih podatkovnih tokov so ponavadi sestavljeni iz 
različnih komponent in modulov in 
s tem zelo zahtevni za postavitev in obvladovanje. 
Potrebnega je veliko specializiranega računalniškega predznanja, 
da se postavi ustrezna računalniška arhitektura, 
ki na učinkovit in hiter način podatke obravnava. 
Hkrati moramo skrbeti, da je delovanje takšnega sistema čimbolj zanesljivo in 
da pride do čim manj izgube podatkov.
Zaradi tega moramo razviti načine samodejnega popravljanja sistema v primerih izpadov.
Ob povečanju uporabe aplikacije in s 
tem pretoka podatkov si ne smemo privoščiti izpada delovanja sistema, 
zato si želimo, da se zna sistem samodejno prilagajati potrebam.

Ker ima danes vse več podjetij potrebo po uporabi lastnih Big Data aplikacij, 
v magistrski nalogi nalogi predstavimo metodologijo in nekaj konceptov,
katere potrebujemo kot ogrodje za postavitev poljubne Big Data aplikacije.
Podjetja lastne Big Data aplikacije najpogosteje postavljajo v računalniški oblak. 
Med vsemi lastnostmi je zaupnost podatkov ključna in 
odloča med postavitvijo Big data aplikacije v javnem (public cloud) ali privatnem oblaku, 
ki je lahko postavljen na lastni (on-prem) ali tuji infrastrukturi (off-prem).
Zato se v magisterski nalogi osredotočamo na najpogostejša načina namestitve,
v javnem ali privatnem oblaku. 
Primer je narejen na rešitvi Gamayun, ki je v lasti podjetja Medius, 
ampak naša metodologija ni vezana na to rešitev.

\subsection{Cilji in prispevki}
\subsection{Struktura magistrske naloge}