\subsection{Arhitekturi Kappa in Lambda}

Arhitekture Big Data so raznolike,
ampak okvirno jih lahko uvrstimo v dva razreda: Kappa ali Lambda arhitekturo~\cite{kappa_lambda}.
Lambda arhitektura vsebuje storitve za procesiranje podatkovnih tokov
in za procesiranje paketov, medtem ko Kappa vsebuje samo storitve za
procesiranje tokov in delo s paketi obravnava kot tokove z zakasnitvijo.

Kappa in Lambda predstavljata dva načina razmišljanja in sicer posploševanje in
specializacija.
Kappa je posplošena, kar pomeni da nam ni treba pisati konfiguracij
za storitve za delo s paketi in ni usklajevanja med storitvami,
ampak zaradi tega je delovanje počasnejše, kot če bi imeli orodje,
ki je specializirano za delo s paketi.
Pri Lambdi je problem ravno obraten, imamo pravo orodje,
ampak moramo usklajevati konfiguracijo med storitvami za tokove in za pakete.